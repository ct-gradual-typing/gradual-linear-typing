\documentclass{article}

\usepackage{amssymb,amsmath, amsthm}
\usepackage{fullpage}
\usepackage{hyperref}
\usepackage{mathpartir}
\usepackage[barr]{xy}
\usepackage{mdframed}
\usepackage{supertabular}
\usepackage{todonotes}
\usepackage{enumitem}
\usepackage{listings}
\usepackage{color}


\usepackage{caption}
\captionsetup[figure]{name=Diagram}

\newtheorem{theorem}{Theorem}
\newtheorem{lemma}[theorem]{Lemma}
\newtheorem{corollary}[theorem]{Corollary}
\newtheorem{definition}[theorem]{Definition}
\newtheorem{proposition}[theorem]{Proposition}
\newtheorem{example}[theorem]{Example}

%% This renames Barr's \to to \mto.  This allows us to use \to for imp
%% and \mto for a inline morphism.
\let\mto\to
\let\to\relax
\newcommand{\to}{\rightarrow}
\newcommand{\ndto}[1]{\to_{#1}}
\newcommand{\ndwedge}[1]{\wedge_{#1}}

% Commands that are useful for writing about type theory and programming language design.
%% \newcommand{\case}[4]{\text{case}\ #1\ \text{of}\ #2\text{.}#3\text{,}#2\text{.}#4}
\newcommand{\interp}[1]{\llbracket #1 \rrbracket}
\newcommand{\normto}[0]{\rightsquigarrow^{!}}
\newcommand{\join}[0]{\downarrow}
\newcommand{\redto}[0]{\rightsquigarrow}
\newcommand{\nat}[0]{\mathbb{N}}
\newcommand{\fun}[2]{\lambda #1.#2}
\newcommand{\CRI}[0]{\text{CR-Norm}}
\newcommand{\CRII}[0]{\text{CR-Pres}}
\newcommand{\CRIII}[0]{\text{CR-Prog}}
\newcommand{\subexp}[0]{\sqsubseteq}
%% Must include \usepackage{mathrsfs} for this to work.

\date{}

\let\b\relax
\let\d\relax
\let\t\relax
\let\r\relax
\let\c\relax
\let\j\relax
\let\wn\relax
\let\H\relax
\let\split\relax
\let\S\relax

% Cat commands.
\newcommand{\powerset}[1]{\mathcal{P}(#1)}
\newcommand{\cat}[1]{\mathcal{#1}}
\newcommand{\func}[1]{\mathsf{#1}}
\newcommand{\iso}[0]{\mathsf{iso}}
\newcommand{\H}[0]{\func{H}}
\newcommand{\J}[0]{\func{J}}
\newcommand{\catop}[1]{\cat{#1}^{\mathsf{op}}}
\newcommand{\Hom}[3]{\mathsf{Hom}_{\cat{#1}}(#2,#3)}
\newcommand{\limp}[0]{\multimap}
\newcommand{\colimp}[0]{\multimapdotinv}
\newcommand{\dial}[1]{\mathsf{Dial_{#1}}(\mathsf{Sets^{op}})}
\newcommand{\dialSets}[1]{\mathsf{Dial_{#1}}(\mathsf{Sets})}
\newcommand{\dcSets}[1]{\mathsf{DC_{#1}}(\mathsf{Sets})}
\newcommand{\sets}[0]{\mathsf{Sets}}
\newcommand{\obj}[1]{\mathsf{Obj}(#1)}
\newcommand{\mor}[1]{\mathsf{Mor(#1)}}
\newcommand{\id}[0]{\mathsf{id}}
\newcommand{\lett}[0]{\mathsf{let}\,}
\newcommand{\inn}[0]{\,\mathsf{in}\,}
\newcommand{\cur}[1]{\mathsf{cur}(#1)}
\newcommand{\curi}[1]{\mathsf{cur}^{-1}(#1)}
\newcommand{\lolli}{\multimap}

\newcommand{\w}[1]{\mathsf{weak}_{#1}}
\newcommand{\c}[1]{\mathsf{contra}_{#1}}
\newcommand{\cL}[1]{\mathsf{contraL}_{#1}}
\newcommand{\cR}[1]{\mathsf{contraR}_{#1}}
\newcommand{\e}[1]{\mathsf{ex}_{#1}}

\newcommand{\m}[1]{\mathsf{m}_{#1}}
\newcommand{\n}[1]{\mathsf{n}_{#1}}
\newcommand{\b}[1]{\mathsf{b}_{#1}}
\newcommand{\d}[1]{\mathsf{d}_{#1}}
\newcommand{\h}[1]{\mathsf{h}_{#1}}
\newcommand{\p}[1]{\mathsf{p}_{#1}}
\newcommand{\q}[1]{\mathsf{q}_{#1}}
\newcommand{\t}[0]{\mathsf{t}}
\newcommand{\r}[1]{\mathsf{r}_{#1}}
\newcommand{\s}[1]{\mathsf{s}_{#1}}
\newcommand{\j}[1]{\mathsf{j}_{#1}}
\newcommand{\jinv}[1]{\mathsf{j}^{-1}_{#1}}
\newcommand{\wn}[0]{\mathop{?}}
\newcommand{\codiag}[1]{\bigtriangledown_{#1}}
\newcommand{\T}[0]{\mathsf{T}}

\newcommand{\split}[0]{\mathsf{split}}
\newcommand{\squash}[0]{\mathsf{squash}}
\newcommand{\bx}[0]{\mathsf{box}}
\newcommand{\Bx}[0]{\mathsf{Box}}
\newcommand{\error}[0]{\mathsf{error}}
\newcommand{\err}[0]{\mathsf{err}}
\newcommand{\unbox}[0]{\mathsf{unbox}}
\newcommand{\Unbox}[0]{\mathsf{Unbox}}

\newenvironment{changemargin}[2]{%
  \begin{list}{}{%
    \setlength{\topsep}{0pt}%
    \setlength{\leftmargin}{#1}%
    \setlength{\rightmargin}{#2}%
    \setlength{\listparindent}{\parindent}%
    \setlength{\itemindent}{\parindent}%
    \setlength{\parsep}{\parskip}%
  }%
  \item[]}{\end{list}}

\newenvironment{diagram}{
  \begin{center}
    \begin{math}
      \bfig
}{
      \efig
    \end{math}
  \end{center}
}

%% %% Ott
%% \input{BiLNL-inc}

\urldef{\mailsa}\path|{heades}@augusta.edu|

\begin{document}

\title{\vspace{-45px}On gradual LNL-models}
\author{Richard Blair and Harley Eades III}
\date{Computer Science, Augusta University\\Computer Science, University of Iowa}

\maketitle 

\section{Closed functors are monoidal}

\begin{definition} A \textbf{monoidal category} is a category $\cat{M}$ equipped with a bifunctor $ - \otimes - : \m \times \cat{M} \mto \cat{M}$, a distinguished object $I$, and natural isomorphisms
  \begin{gather}
    \alpha_{X,Y,Z} : (X \otimes Y) \otimes Z \mto X \otimes (Y \otimes Z) \\
    l_{X} : I \otimes X \mto X \\
    r_{X} : X \otimes I \mto X
  \end{gather}
  for which the following two coherence diagrams commute:
\begin{center}
  \begin{math}
    \bfig
    \node botleft(0,0)[((W \otimes X) \otimes Y)]
    \node midleft(0,900)[(W \otimes X) \otimes (Y \otimes Z)]
    \node topleft(0,1800)[((W \otimes X) \otimes Y) \otimes Z]
    \node botright(1800,0)[W \otimes ((X \otimes Y) \otimes Z)]
    \node topright(1800,1800)[(W \otimes (X \otimes Y)) \otimes Z]
    \arrow[topleft`midleft; \alpha]
    \arrow[midleft`botleft; \alpha]
    \arrow|b|/<-/[botleft`botright; 1 \, \otimes \, \alpha]
    \arrow[topleft`topright; \alpha \, \otimes \, 1]
    \arrow|r|[topright`botright; \alpha]
    \efig
  \end{math}
  \vspace{15px}
  \begin{math}
    \bfig
    \Vtriangle[(X \otimes I) \otimes Y`X \otimes (I \otimes Y)`X \otimes Y;\alpha`r \, \otimes \, 1`1 \, \otimes \: l]
    \efig
  \end{math}
\end{center}
and for which $l_I = r_I$.
\end{definition}

\begin{definition}
A \textbf{closed monoidal category} is a monoidal category $(\cat{M} \otimes, l, I, \alpha, l, r)$ such that for each $B \in \cat{M}_0$ the functor $- \otimes B : \cat{M} \mto \cat{M}$ has a specified right adjoint; that is, for every $A, C \in \cat{M}_0$ an object $(B \lolli C)$ and a natural bijection $\cat{M}(A \otimes B, C) \cong \cat{M}(A, B \lolli C)$.
\end{definition}

\begin{definition} Given monoidal categories $(\cat{M}, \otimes, I, \alpha, l, r)$ and $(\cat{M}^\prime, \otimes^\prime, I^\prime, \alpha^\prime, l^\prime, r^\prime)$ \textbf{monoidal functor} $F : \cat{M} \mto \cat{M}^\prime$ is a functor from $\cat{M}$ to $\cat{M}^\prime$ equipped with a map $m_I : I^\prime \mto F(I)$ in $\cat{M}^\prime$ and a natural transformation $m_{X,Y} : F(X) \otimes^\prime F(Y) \mto F(X \otimes Y)$ satisfying the following coherence conditions:
  \begin{center}
    \begin{math}
      \bfig
        \node botleft(0,0)[F((X \otimes Y) \otimes Z)]
        \node midleft(0,900)[F(X \otimes Y) \otimes^\prime F(Z)]
        \node topleft(0,1800)[(F(X) \otimes^\prime F(Y)) \otimes^\prime F(Z)]
        \node botright(1800,0)[F(X \otimes (Y \otimes Z))]
        \node midright(1800,900)[F(X) \otimes^\prime F(Y \otimes Z)]
        \node topright(1800,1800)[F(X) \otimes^\prime (F(Y) \otimes^\prime F(Z))]
        \arrow[topleft`midleft; m \, \otimes^\prime \, 1]
        \arrow[midleft`botleft; m]
        \arrow|b|[botleft`botright; F(\alpha)]
        \arrow[topleft`topright; \alpha^\prime]
        \arrow|r|[topright`midright; 1 \, \otimes^\prime\, m]
        \arrow|r|[midright`botright; m]
        \efig
    \end{math}
  \end{center}

  \begin{center}
    \begin{math}
      \bfig
      \square(0,0)/>`>`<-`>/<800,800>[I^\prime \otimes^\prime F(X)`F(X)`F(I)\otimes^\prime F(X)`F(I \otimes X);l^\prime`m \, \otimes^\prime \, 1`F(l)`m]
      \efig
      \bfig
      \square(0,0)/>`>`<-`>/<800,800>[F(X) \otimes^\prime I^\prime`F(X)`F(X)\otimes^\prime F(I)`F(X \otimes I);r^\prime`1 \, \otimes^\prime \, m`F(r)`m]
      \efig
    \end{math} 
  \end{center}
\end{definition}

\begin{definition} Given closed monoidal categories $(\cat{M}, \otimes, \Rightarrow, l, I, \alpha, l)$ and $(\cat{M}^\prime, \otimes^\prime, \Rightarrow^\prime, l^\prime, I^\prime, \alpha, l, r)$, a \textbf{closed functor} $F : \cat{M} \mto \cat{M}^\prime$ is a functor from $\cat{M}$ to $\cat{M}^\prime$ equipped with a map $\mu_I : I^\prime \mto F(I)$ and a natural transformation $\phi_{X,Y} : F(X \Rightarrow Y) \mto F(X) \Rightarrow^\prime F(Y)$ such that the following coherence conditions hold:

  \begin{center}
    \begin{math}
      \bfig
      \square(0,0)<800,800>[F(I)`F(X \Rightarrow X)`I^\prime`F(X) \Rightarrow^\prime F(X);F(\cur{\l_X})`\mu_I`\phi_{X,Y}`\cur{l_X}]
      \efig
      \bfig
      \square(0,0)<800,800>[F(I \Rightarrow X)`F(I) \Rightarrow^\prime F(X)`F(X)`I^\prime \Rightarrow^\prime F(X); \phi_{I,X} ` F(\cur{r_X}^{-1}) ` \mu_I \Rightarrow^\prime 1 `\cur{r_{F(X)}^{\prime}}]
      \efig
    \end{math}
  \end{center}
  
  
\end{definition}
\vspace{-15px}

\begin{theorem} A closed functor $(F, \mu_I, \phi_{X,Y})$ between closed monoidal categories $(\cat{M}, \otimes, \Rightarrow, l, I, \alpha, l)$
  and $(\cat{M}^\prime, \otimes^\prime, \Rightarrow^\prime, l^\prime, I^\prime, \alpha^\prime, l^\prime, r^\prime)$ is a monoidal functor, taking $m_I = \mu_I$ and $m_{X,Y} = \curi{F(\cur{id_{X \otimes Y}});\phi_{Y, X \otimes Y}}$.
\end{theorem}
\begin{proof}
  To come...
\end{proof}

\nocite{*}
\bibliographystyle{plainurl}
\bibliography{ref}

\end{document}

%%% Local Variables: 
%%% mode: latex
%%% TeX-master: t
%%% End: 
