\documentclass{article}
\usepackage{amssymb,amsmath}
\usepackage{amsthm}
\usepackage{mathtools}
\usepackage{cmll}
\usepackage{txfonts}
\usepackage{graphicx}
\usepackage{stmaryrd}
\usepackage{todonotes}
\usepackage{mathpartir}
\usepackage{hyperref}
\usepackage{mdframed}
\usepackage[barr]{xy}
\usepackage{comment}
\usepackage{graphicx}
\usepackage[inline]{enumitem}

\usepackage{caption}
\captionsetup[figure]{name=Diagram}

\newtheorem{theorem}{Theorem}
\newtheorem{lemma}[theorem]{Lemma}
\newtheorem{corollary}[theorem]{Corollary}
\newtheorem{definition}[theorem]{Definition}
\newtheorem{proposition}[theorem]{Proposition}
\newtheorem{example}[theorem]{Example}

%% This renames Barr's \to to \mto.  This allows us to use \to for imp
%% and \mto for a inline morphism.
\let\mto\to
\let\to\relax
\newcommand{\to}{\rightarrow}
\newcommand{\ndto}[1]{\to_{#1}}
\newcommand{\ndwedge}[1]{\wedge_{#1}}

% Commands that are useful for writing about type theory and programming language design.
%% \newcommand{\case}[4]{\text{case}\ #1\ \text{of}\ #2\text{.}#3\text{,}#2\text{.}#4}
\newcommand{\interp}[1]{\llbracket #1 \rrbracket}
\newcommand{\normto}[0]{\rightsquigarrow^{!}}
\newcommand{\join}[0]{\downarrow}
\newcommand{\redto}[0]{\rightsquigarrow}
\newcommand{\nat}[0]{\mathbb{N}}
\newcommand{\fun}[2]{\lambda #1.#2}
\newcommand{\CRI}[0]{\text{CR-Norm}}
\newcommand{\CRII}[0]{\text{CR-Pres}}
\newcommand{\CRIII}[0]{\text{CR-Prog}}
\newcommand{\subexp}[0]{\sqsubseteq}
%% Must include \usepackage{mathrsfs} for this to work.

\date{}

\let\b\relax
\let\d\relax
\let\t\relax
\let\r\relax
\let\c\relax
\let\j\relax
\let\wn\relax
\let\H\relax
\let\split\relax
\let\S\relax

% Cat commands.
\newcommand{\powerset}[1]{\mathcal{P}(#1)}
\newcommand{\cat}[1]{\mathcal{#1}}
\newcommand{\func}[1]{\mathsf{#1}}
\newcommand{\iso}[0]{\mathsf{iso}}
\newcommand{\H}[0]{\func{H}}
\newcommand{\J}[0]{\func{J}}
\newcommand{\catop}[1]{\cat{#1}^{\mathsf{op}}}
\newcommand{\Hom}[3]{\mathsf{Hom}_{\cat{#1}}(#2,#3)}
\newcommand{\limp}[0]{\multimap}
\newcommand{\colimp}[0]{\multimapdotinv}
\newcommand{\dial}[1]{\mathsf{Dial_{#1}}(\mathsf{Sets^{op}})}
\newcommand{\dialSets}[1]{\mathsf{Dial_{#1}}(\mathsf{Sets})}
\newcommand{\dcSets}[1]{\mathsf{DC_{#1}}(\mathsf{Sets})}
\newcommand{\sets}[0]{\mathsf{Sets}}
\newcommand{\obj}[1]{\mathsf{Obj}(#1)}
\newcommand{\mor}[1]{\mathsf{Mor(#1)}}
\newcommand{\id}[0]{\mathsf{id}}
\newcommand{\lett}[0]{\mathsf{let}\,}
\newcommand{\inn}[0]{\,\mathsf{in}\,}
\newcommand{\cur}[1]{\mathsf{cur}(#1)}
\newcommand{\curi}[1]{\mathsf{cur}^{-1}(#1)}
\newcommand{\lolli}{\multimap}

\newcommand{\w}[1]{\mathsf{weak}_{#1}}
\newcommand{\c}[1]{\mathsf{contra}_{#1}}
\newcommand{\cL}[1]{\mathsf{contraL}_{#1}}
\newcommand{\cR}[1]{\mathsf{contraR}_{#1}}
\newcommand{\e}[1]{\mathsf{ex}_{#1}}

\newcommand{\m}[1]{\mathsf{m}_{#1}}
\newcommand{\n}[1]{\mathsf{n}_{#1}}
\newcommand{\b}[1]{\mathsf{b}_{#1}}
\newcommand{\d}[1]{\mathsf{d}_{#1}}
\newcommand{\h}[1]{\mathsf{h}_{#1}}
\newcommand{\p}[1]{\mathsf{p}_{#1}}
\newcommand{\q}[1]{\mathsf{q}_{#1}}
\newcommand{\t}[0]{\mathsf{t}}
\newcommand{\r}[1]{\mathsf{r}_{#1}}
\newcommand{\s}[1]{\mathsf{s}_{#1}}
\newcommand{\j}[1]{\mathsf{j}_{#1}}
\newcommand{\jinv}[1]{\mathsf{j}^{-1}_{#1}}
\newcommand{\wn}[0]{\mathop{?}}
\newcommand{\codiag}[1]{\bigtriangledown_{#1}}
\newcommand{\T}[0]{\mathsf{T}}

\newcommand{\split}[0]{\mathsf{split}}
\newcommand{\squash}[0]{\mathsf{squash}}
\newcommand{\bx}[0]{\mathsf{box}}
\newcommand{\Bx}[0]{\mathsf{Box}}
\newcommand{\error}[0]{\mathsf{error}}
\newcommand{\err}[0]{\mathsf{err}}
\newcommand{\unbox}[0]{\mathsf{unbox}}
\newcommand{\Unbox}[0]{\mathsf{Unbox}}

\newenvironment{changemargin}[2]{%
  \begin{list}{}{%
    \setlength{\topsep}{0pt}%
    \setlength{\leftmargin}{#1}%
    \setlength{\rightmargin}{#2}%
    \setlength{\listparindent}{\parindent}%
    \setlength{\itemindent}{\parindent}%
    \setlength{\parsep}{\parskip}%
  }%
  \item[]}{\end{list}}

\newenvironment{diagram}{
  \begin{center}
    \begin{math}
      \bfig
}{
      \efig
    \end{math}
  \end{center}
}

%% %% Ott
%% \input{BiLNL-inc}

\urldef{\mailsa}\path|{heades}@augusta.edu|

\begin{document}

\title{\vspace{-45px}On gradual LNL-models}
\author{Richard Blair and Harley Eades III}
\date{Computer Science, Augusta University\\Computer Science, University of Iowa}

\maketitle 

\section{Gradual LNL model}

\begin{definition} A \textbf{gradual LNL-model} consists of
  \begin{enumerate}
  \item a gradual $\lambda$-model $G\lambda^\ast = (\cat{T}, \cat{C}, \ast_{\cat{C}}, \T, \split, \squash, \bx, \unbox, \error)$
  \item a symmetric monoidal closed category $(\cat{L}, I, \otimes, \lolli)$ with distinguished object $\ast_{\cat{L}}$
  \item a pair of strong closed monoidal functors $(G, n) : \cat{L} \mto \cat{C}$ and $(F, \hat{F}) : \cat{C} \mto \cat{L}$ forming a monoidal adjunction with $F \dashv G$.
  \end{enumerate}
  
\end{definition}

\begin{theorem}
  A gradual LNL-model is an LNL model.
\end{theorem}
   
The subscript on $\ast$ shall be omitted below where the meaning is clear from the context.

We now prepare to show that $FG$ preserves the retract properties of $\squash_S$ and $\split_S$ in $\cat{C}$ for $\ast \Rightarrow \ast$ and $\ast \times \ast$. That is, we shall show that $FG(\ast) \lolli FG(\ast)$ is a retract of $FG\ast$, as is $FG\ast \otimes FG\ast$.

\begin{lemma}
  $FG(\ast) \lolli FG(\ast)$ is a retract of $FG\ast$ with morphisms
  \[
  \cat{L}\squash_{FG\ast \lolli FG\ast} \coloneqq \hat{F}_{\ast,\ast} ; F(\Unbox_{G\ast} \Rightarrow \Box{G\ast}); F \squash_{\ast \Rightarrow \ast} ; F(\Unbox_{G\ast})
  \]
  and
  \[
  \cat{L}\split_{FG \ast \lolli FG \ast} \coloneqq F(\Bx_{G\ast}) ; F\split_{\ast \Rightarrow \ast} ; F(\Bx_{G\ast} \Rightarrow \Unbox_{G\ast}) ;\hat{F}^{-1}_{\ast,\ast}
  \]
  % This doesn't work.
\end{lemma}
\begin{proof}
  Observe that:

  \[
  \setlength{\arraycolsep}{2px}
  \begin{array}{lll}
    \cat{L}\squash ; \cat{L}\split
    & = & m_{\ast,\ast} ; F \squash_{\ast \Rightarrow \ast} ; F \split_{\ast \Rightarrow \ast} ; p_{\ast,\ast} \\
   
  \end{array}
  \]

\end{proof}
\vspace{-15px}

\nocite{*}
\bibliographystyle{plainurl}
\bibliography{ref}

\end{document}

%%% Local Variables: 
%%% mode: latex
%%% TeX-master: t
%%% End: 
